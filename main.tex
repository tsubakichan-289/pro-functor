\documentclass[dvipdfmx]{jsarticle}
\usepackage{tikz}
\usepackage{amsmath}
\usepackage{amssymb}
\usetikzlibrary{positioning}

\usepackage{amsfonts}
\usepackage[bbgreekl]{mathbbol}

\newcommand{\yslant}{0.4}
\newcommand{\xslant}{-0.6}

\begin{document}
集合をスカラーという.

函手$(F:\mathbb{C}\rightarrow\mathbb{Set})$をベクトルと呼び,$F_\mathbb{C}$と表す.

函手$(G:\mathbb{D^{op}}\rightarrow\mathbb{Set})$をコベクトルと呼び,$G^\mathbb{D}$と表す.

双函手$(H:\mathbb{D^{op}\times\mathbb{C}}\rightarrow\mathbb{Set})$を行列と呼び,$H^\mathbb{D}_\mathbb{C}$と表す.

$F_\mathbb{C}$に$c\in \mathbb{C}$を代入したものは集合となり,$F_{(c)}$と表す.

$G^\mathbb{D}$に$d\in \mathbb{D}$を代入したものは集合となり,$G^{(d)}$と表す.

$H_\mathbb{C}^\mathbb{D}$に$c\in \mathbb{C}$を代入したものはコベクトルとなり,$H_{(c)}^\mathbb{D}$と表す.

$H_\mathbb{C}^\mathbb{D}$に$d\in \mathbb{D}$を代入したものはベクトルとなり,$H_\mathbb{C}^{(d)}$と表す.

ある集合$S$を$\hat{S}:\mathbb{1}\rightarrow\mathbb{Set}$として考え,$\hat{S}(*)=S$とすると$\hat{S}=\hat{S}_\mathbb{1}=\hat{S}^\mathbb{1}$であり,$S=\hat{S}_{(*)}=\hat{S}^{(*)}$と表すことができる,

行列$P_\mathbb{X}^\mathbb{X}$のエンドを$\displaystyle\int P_\mathbb{X}^\mathbb{X} d\mathbb{X}$と表す.

行列$P_\mathbb{X}^\mathbb{X}$のコエンドを$\displaystyle\int {P_\mathbb{X}^\mathbb{X}}^{d\mathbb{X}}$と表す.

米田埋め込み$\text{Hom}(A,B)$を$\Delta^{(A)}_{(B)}$と表す.

添え字がついた文字をテンソルという.つまりスカラー,ベクトルやコベクトル,行列はテンソルである.


\end{document}